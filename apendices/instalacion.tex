%%%%%%%%%%%%%%%%%%%%%%%%%%%%%%%%%%%
%%%%%%%%%%%%%%%%%%%%%%%%%%%%%%%%%%%
\chapter{Instalación do sistema Caronte}

Neste apéndice detallaranse as principais fases para a instalación e posta a punto do noso sistema Caronte.
En primeiro lugar, indicaranse os requisitos do sistema. A continuación, os pasos para a súa instalación e despregue, tanto do servidor como da aplicación Android.
Por último, remataremos co manual do usuario e unha guía rápida de funcionamento.

%%%%%%%%%
\section{Requisitos}

\subsection{Servidor Caronte}
\subsection{Aplicación Android: Caronte}
Para poder instalar e utilizar a aplicación, requírese un dispositivo móbil que cumpra unha serie de requisitos software, pero sobre todo hardware. A lista de necesidades pódese ver a continuación:

\begin{itemize}
	\item Sistema operativo Android superior ou igual á versión 4.4 (Kit Kat).
	\item Sensor GPS.
	\item Conexión Wi-Fi.
	\item Conexión datos.
	\item Bluetooth (recomendado).
	\item 1,5 GB de memoria RAM para un uso .
	\item 7 MB libres de espazo no dispositivo. Máis se se desexan visualizar imaxes de puntos de interese.
\end{itemize}

%%%%%%%%%
\section{Instalación}

\subsection{Servidor Caronte}
\subsection{Aplicación Android: Caronte}
A instalación da aplicación pode levarse a cabo a través da tenda oficial de Google: Google Play. Non obstante, tamén se pode proceder á instalación manual do arquivo APK se se habilita a instalación de aplicacións con orixe descoñecida nos axustes de seguridade do teléfono.

Non hai ningún tipo de necesidade a maiores no momento da instalación, xa que a petición de permisos necesarios para o funcionamento da aplicación solicítanse dinamicamente na súa execución.


%%%%%%%%%
\section{Manuales de usuario}

Nesta sección comentaranse os manuais de usuario para os tres actores do sistema Caronte:
administrador, xestor de contidos e usuario anónimo.

\subsection{Administrador do servidor Caronte}
despliegue, inicialización

cambio de rol

\subsection{Aplicación Caronte para usuarios anónimos}

vistas e guía rápida de funcionamento

\subsection{Apicación Caronte para xestores de contidos}

usuario autenticado

usuario con permisos de edición: POI, recorridos, etc.

