\chapter*{Resumo}

Neste proxecto desenvolvemos o sistema Caronte para o guiado no interior dos museos. A plataforma consta de tres módulos: un servidor de datos, un servidor para localización en interiores e unha aplicación Android na que se mostra toda a información. O intercambio de datos está securizada e realízase a través de servizos REST en formato JSON.

O servidor almacena nunha base de datos relacional (PostgresSQL) a posición e a información relevante dos puntos de interese (POIs). Tamén almacena os percorridos de interese e os datos dos usuarios con privilexios. Está despregado na nube, concretamente en Amazon Web Servizes (AWS), aínda que valería calquera outra plataforma. 

Unha funcionalidade única do noso sistema é a localización en interiores. Empregamos a infraestrutura de Situm Technologies que ten unha precisión de ata 2 metros. A partir da posición coñecemos se o usuario está preto dun POI e calculamos a ruta para ir a un POI ou seguir un percorrido. Tamén nos permite saber os tempos e os percorridos reais das visitas.

A aplicación Android está dispoñíbel en Google Play e ten dous propósitos. En primeiro lugar, pódese instalar de forma anónima e cando está preto do museo mostra a información almacenada: POIs, percorridos, tempos estimados de visita e desprazamentos, imaxes, etc. En segundo lugar, os usuarios con permisos poden crear e editar os contidos dende a propia aplicación, o que facilita a xeolocalización e a captura de imaxes.

Como resultado final podemos destacar a boa aceptación da aplicación e a ausencia de problemas durante as probas. A plataforma creada é fácil de despregar e manter, robusta e moi escalábel. Ao estar na nube os recursos adáptanse ao uso que se fai deles. 

\section*{Palabras clave}

Guiado en museos; Localización en interiores; Situm Technologies; Android.
