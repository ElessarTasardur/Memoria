\chapter{Conceptos teóricos}

Neste capítulo farase unha relación dos conceptos teóricos nos que se basea o proxecto.


\section{Android}

É un sistema operativo principalmente dirixido a dispositivos móbiles que ten como base o núcleo Linux. Nos seus inicios foi desenvolvido pola empresa Android Inc., financiada por Google, multinacional estadounidense que acabaría por se facer co seu control. É o sistema operativo con maior cuota de mercado dentro dos dispositivos móbiles cunha ampla marxe sobre a súa competencia.

Escóllese este sistema operativo por ser de código aberto, facilitar a programación sobre el e por ter unha base de usuarios maior.


\section{GPS}

O Global Positioning System (GPS) ou Sistema de Posicionamento Global en galego, é un sistema de navegación por satélite utilizado en todo o mundo. Foi creado e é mantido polo goberno dos Estados Unidos, tendo unha orixe militar. Funciona grazas a 27 satélites (24 principais e 3 de respaldo) que se atopan en órbita sobre o planeta. Permiten a localización dun dispositivo en calquera punto do globo. O posicionamento lógrase mediante triangulación cando se consigue a conexión con, como mínimo, catro satélites. Non é preciso o uso de redes telefónicas para estas conexións xa que non é preciso que o usuario envíe ningún tipo de información.

Os satélites emiten un sinal de xeito continuo, mais este sinal non é moi potente polo que pode verse afectado por obstáculos, tales como dificultades orográficas ou edificios, polo cal non é posíbel utilizalo dentro de calquera tipo de edificio.


\section{Posicionamento en interiores}

O impedimento de utilizar o GPS en interiores provocou que se buscasen maneiras de permitir un posicionamento para dispositivos móbiles en entornos baixo teito. Para logralo utilízase calquera tipo de información recollida polo dispositivo, tales coma ondas de radio, campos magnéticos ou sinais acústicos. Para unha maior efectividade destes sistemas pódense dispoñer de diversos elementos emisores de ondas en puntos estratéxicos coma poden ser os beacons: dispositivos emisores de baixo consumo que utilizan a tecnoloxía Bluetooth.


\section{Servizos web}

Son un conxunto de protocolos e estándares que permiten o intercambio de datos entre distintas aplicacións. A comunicación non se ve afectada polas linguaxes nas que se escriben esas aplicacións nin polas plataformas nas que estas se executan, polo que non é preciso coñecer como están feitos.

\subsection{Servizos web REST}

REST (REpresentational State Transfer - transferencia de estado figurativo) é un estilo de arquitectura software para sistemas distribuídos que require unha comunicación cliente/servidor sen estado e cacheable a través dunha interface uniforme entre compoñentes.

Neste estilo arquitectónico, os datos e as funcionalidades utilizadas son consideradas recursos e accédese a eles mediante Identificadores de Recurso Uniformes (Uniform Resource Identifiers - URI), que tradicionalmente se identifican coas ligazóns na Web. Os recursos son utilizados mediante un conxunto de operacións simples e ben definidas. A arquitectura utilizada é de tipo cliente/servidor e está deseñada para utilizar un protocolo de comunicacións sen estado, tipicamente HTTP. En REST, os clientes e os servidores intercambian representacións de recursos utilizando unha interface e un protocolo estandarizados. Os principios que perseguen as aplicacións REST son a simplicidade, a lixeireza e a rapidez.