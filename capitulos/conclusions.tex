\chapter{Conclusións e traballo futuro}
Este capítulo está adicado a resumir as principais conclusión deste proxecto así coma o traballo futuro.


\section{Conclusións}

Neste traballo proponse unha plataforma para a guía de museos que consta de dúas partes: un servidor e unha aplicación Android.
Ademais, empregamos o SDK e os servidores de Situm Technologies S.L. para determinar a posición dentro dun edificio.

As principais características da nosa plataforma son:
\begin{itemize}
	\item Modelo de datos flexible: puntos de interese (POI) e percorridos de visita (lista de POIs). Tempo estimado e outros datos de interese como poden ser imaxes.
	\item Autenticación de usuarios para realizar tarefas de mantemento e edición de contidos.
	\item Usuarios anónimos para facilitar o uso da aplicación.
	\item Servidor na nube: procura escalabilidade, facilita o mantemento, permite axuste automático dos recursos empregados ao uso que se fai deles, etc.
	\item Aplicación Android intuitiva e fácil de empregar para mostrar información do museo: mapas de cada planta, POIs, percorridos de visita, etc.
	\item Localización Situm para calcular a posición dentro dos museos (implica unha calibración previa para cada planta dos edificios, por parte do administrador da plataforma ou do xestor de contidos do museo).
	\item Autenticación vía Google para identificación dos usuarios privilexiados do noso sistema.
	\item Edición de datos (POI e percorridos de visita) dentro da propia aplicación Android.
\end{itemize}


\section{Traballo futuro}

Este proxecto resolve os requisitos formulados inicialmente, pero tamén abre novas vías e retos para o futuro:

\begin{itemize}
	\item Determinar automaticamente onde están máis tempo os visitantes no museo. Desta maneira poderíanse axustar mellor os modelos: o tempo para ir dun POI a outro, o tempo adicado a un POI, as rutas máis empregadas, os POIs máis visitados, a orde na que se visitan, etc.
	\item Permitir aos xestores de contido engadir máis información referente aos puntos de interese e aos percorridos, facendo máis rica a experiencia de uso para os usuarios.
	\item Outro punto interesante sería determinar, grazas ao seus movementos, que tipos de usuarios visitan o museo. Con estes datos, poderíase "adaptar" a información que se mostra a cada usuario.
\end{itemize}
