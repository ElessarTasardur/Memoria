\chapter{Introdución}

Neste primeiro capítulo explicaremos brevemente os aspectos básicos do proxecto: contexto xeral do problema a solventar, solucións existentes na actualidade e a nosa proposta para o proxecto. No último apartado deste capítulo explicaremos a estrutura da memoria.

\section{Contexto}

Os dispositivos móbiles intelixentes (smartphones) revolucionaron a tecnoloxía nestes últimos anos. Pódense ter todo tipo de aplicacións e utilidades ao alcance da nosa man e gardalas nun peto, cun tamaño semellante ao que pode ter unha carteira. Grazas a estes dispositivos podemos realizar todo tipo de accións que antes estaban limitadas a unha soa máquina ou aparello: sustituíron calculadoras, libretas, lanternas, mapas... É precisamente na sustitución dos mapas onde se quere centrar este proxecto. De todos é coñecida a utilidade das aplicacións baseadas na xeolocalización para o guiado e posicionamento en lugares descoñecidos nos que pisamos por primeira vez, xa non é unha aventura viaxar a unha cidade sen coñecela previamente; o único que se precisa é un móbil intelixente e activar o servizo GPS do mesmo para poder percorrer a cidade sen se perder. Mais o GPS ten, entre outras, unha gran limitación: non poden ser utilizados no interior de edificios. Esta capacidade tamén sería moi útil se se puidese utilizar en certos edificios cun gran tamaño e ao que un posíbel usuario non estea acostumado. Recentemente, producíronse grandes avances na determinación fiábel e precisa da posición dun teléfono móbil en interiores, polo que xa se pode solventar esa restrición do GPS.

Son múltiples os casos nos que pode resultar útil un sistema de guiado nun edificio, xa sexa público ou privado. Entre eles podemos destacar centros médicos, nos que poder guiar a pacientes á súa consulta sen necesidade de solicitar axuda; centros comerciais, para situar as tendas do seu interior; ou museos, nos que poder ofrecer a localización das obras da súa colección. Ésta última opción foi a seleccionada para este proxecto xa que ten engadidos máis interesantes como a creación de percorridos para que os usuarios visiten o museo sengundo os seus gustos ou o tempo dispoñíbel.

Para a realización do proxecto seleccionouse o sistema de posicionamento en interiores de Situm debido ao seu bo funcionamento e á súa accesibilidade.

\section{Solucións actuais}

A continuación enumeramos distintas solucións para o guiado e obtención de información dentro de museos. Imos dende as máis estendidas e básicas ata as máis elaboradas cun maior grao de semellanza coa nosa idea de proxecto.
\begin{itemize}
	\item Guías en papel: os típicos mapas con breves explicacións existentes en todos os museos. Precísanse paneis ao longo do museo para o posicionamento. Non hai ningunha interacción.
	\item Audioguías: configurábeis por idioma. É necesario introducir o código asignado a un elemento para escoitar un audio sobre el.
	\item Comezan a estenderse as guías multimedia en dispositivos intelixentes debido ás súas grandes posibilidades:
	\begin{itemize}
		\item Solucións propias para cada museo: non teñen tantas opcións como as xenéricas pero compénsano cunha maior especialización e adaptación ao museo en cuestión. O museo do Prado é un exemplo deste tipo de solucións.
		\item ATS Heritage: empresa que crea guías multimedia xenéricas para museos. As guías poden estar dispoñíbeis para iOS, Android e Windows Phone. Teñen a capacidade de localizar no exterior mais non no interior \cite{atsHeritage}.
		\item OrpheoGroup: fabrican dispositivos multimedia propios. O máis configurábel consiste nun teléfono móbil intelixente para o cal se pode preparar contido mediante software propio da empresa.
		\item AcousticGuide: elaboran hardware propio. Permite a localización en interior.
	\end{itemize}
\end{itemize}

\section{Obxectivos}

O obxectivo deste proxecto é desenvolver un sistema que permita a localización e guiado dentro dun edificio (museo), cunha inversión reducida e sen infraestrutura específica, a través dun dispositivo móbil. A aplicación permitirá a localización do usuario ou de puntos de interese, e permitirá unir eses puntos segundo percorridos semanticamente relacionados.

Os obxectivos específicos son:
\begin{itemize}
	\item Localizar un usuario dentro do edificio.
	\item Localizar un punto de interese dentro do edificio.
	\item Permitir o guiado do usuario ata un punto de interese.
	\item Permitir o guiado do usuario entre distintos puntos mediante percorridos.
	\item Proporcionar todas as ferramentas precisas para que un usuario que teña que administrar o edificio poida realizalo comodamente de xeito sinxelo.
\end{itemize}

\section{Estrutura da memoria}

A memoria divídese en dez capítulos intentando conseguir unha división detallada de cada paso da elaboración do proxecto. Os primeiros puntos serven para dar un contexto ao proxecto, sen atender aínda á súa elaboración. No capítulo de Conceptos teóricos faise un pequeno resumo explicativo dos aspectos xerais tratados no proxecto. Conceptos coma o posicionamento en interiores serán explicados neste punto. No terceiro capítulo, Fundamentos tecnolóxicos, revísanse os elementos utilizados na creación do proxecto, tanto dispositivos hardware coma software: smartphones, entornos de desenvolvemento e aplicacións varias. No cuarto punto, Metodoloxía de desenvolvemento, explícanse as condicións nas que se elaborou o proxecto.

É a partir do quinto punto, Análise, onde se comeza a entrar no detalle do mesmo. Descríbense os requisitos que debe cumprir o proxecto e os casos de uso que dan conta deles. O sexto punto correspóndese coa Planificación e custos, onde se detallan os pasos que se deron na elaboracion do proxecto xunto cos custos desglosados.

No sétimo punto faise fincapé no deseño da aplicación, tanto da arquitectura de todos os sistemas involucrados, explicando os motivos desas eleccións e en como están montados coma solucións; coma dos detalles da base de datos e do servizo web.
No seguinte capítulo explícase o proceso de Implementación da aplicación Android e do servizo web.

Os dous últimos capítulos son os de Probas, onde se comproban as funcionalidades do sistema, e o de Conclusións e traballo futuro, onde se reflexiona sobre os obxectivos realmente logrados unha vez rematado.

Finalmente, remátase a memoria en varios apéndices.
