\chapter{Conclusións e traballo futuro}
Este capítulo está adicado a resumir as principais conclusión deste proxecto así coma o traballo futuro.


\section{Conclusións}

Neste traballo proponse unha plataforma para a guía de museos que consta de dúas partes: un servidor e unha aplicación Android.
Ademais, empregamos o SDK e os servidores de Situm Technologies S.L. para determinar a posición dentro dun edificio.

As principais características da nosa plataforma son:
\begin{itemize}
	\item Modelo de datos flexible: puntos de interese (POI) e percorridos de visita (lista de POIs). Tempo estimado e outros datos de interese como poden ser imaxes.
	\item Autenticación de usuarios para realizar tarefas de mantemento e edición de contidos.
	\item Usuarios anónimos para facilitar o uso da aplicación.
	\item Servidor na nube: procura escalabilidade, facilita o mantemento, permite axuste automático dos recursos empregados ao uso que se fai deles, etc.
	\item Aplicación Android intuitiva e fácil de empregar para mostrar información do museo: mapas de cada planta, POIs, percorridos de visita, etc.
	\item Localización Situm para calcular a posición dentro dos museos (implica unha calibración previa para cada planta dos edificios, por parte do administrador da plataforma ou do xestor de contidos do museo).
	\item Autenticación vía Google para identificación dos usuarios privilexiados do noso sistema.
	\item Edición de datos (POI e percorridos de visita) dentro da propia aplicación Android.
\end{itemize}


\section{Traballo futuro}

Este proxecto resolve os requisitos formulados inicialmente, pero tamén abre novas vías e retos para o futuro:

\begin{itemize}
	\item O principal punto a tratar nun futuro debería ser a creación dun cliente web que permita a administración do sistema e a inserción de datos sobre os puntos de interese e os percorridos por parte dos xestores de contido. É innegábel a potencia e a comodidade de crear puntos directamente sobre o mapa, pero a experiencia dos xestores á hora de engadir información sobre os elementos directamente dende o dispositivo móbil non é moi amigábel. Tamén é innegábel a incomodidade dos administradores do sistema de realizar todas as configuracións a través dun sistema xestor de base de datos.
	\item Facer máis atractiva a interface da aplicación móbil, é demasiado simple. Mellorar a información amosada nos puntos de interese e nos percorridos, permitindo a visualización de ligazóns, vídeos ou outro contido multimedia grazas a anotacións HTML.
	\item Determinar automaticamente onde están máis tempo os visitantes no museo. Desta maneira poderíanse axustar mellor os tempos almacenados en base de datos e observar se sería preciso crear engadir algún punto de interese aos percorridos utilizados. Tamén se poderían axustar automaticamente o tempo necesario para a realización dos percorridos e o tempo adicado a cada punto de interese.
	\item Identificar as rutas máis empregadas e ordenalas por maior interese para que poidan ser utilizadas polos usuarios menos expertos.
	\item Controlar a orde na que os usuarios realizan visitas para a creación de percorridos en base a esas rutas.
	\item Permitir aos xestores de contido engadir máis información referente aos puntos de interese e aos percorridos, facendo máis rica a experiencia de uso para os usuarios.
	\item Outro punto interesante sería permitir a cada usuario que indicase certos datos persoais para adaptar os datos dos puntos de interese e dos percorridos. Amosar máis información de carácter profesional a un licenciado en belas artes, en contraposición a unha información máis simple e accesíbel para un estudante de primaria que accede por primeira vez a un museo.
	\item Integrar unha compoñente social dentro do sistema, permitindo aos usuarios a valoración dos percorridos e dos puntos de interese, e incluso a creación de POIs e percorridos propios nun espazo distinto aos creados polos xestores de contido dos museos, para diferencialos dos oficiais.
\end{itemize}
