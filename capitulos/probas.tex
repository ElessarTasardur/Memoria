\chapter{Probas}

Neste capítulo detallaremos as probas realizadas para validar o noso sistema, tanto ao finalizar cada sprint coma ao rematar o desenvolvemento. Tamén se terán en conta aquelas ferramentas utilizadas para localizar e ter un maior control sobre os problemas da aplicación.

\todo[inline]{ver TFG de Gonzalo García y Dylan Lema}


\section{Probas de validación de cada sprint}
Ao longo de todo o desenvolvemento do proxecto realizáronse probas ao finalizar cada sprint, ca fin de identificar posíbeis problemas introducidos sexa sobre as funcionalidades novas ou nas xa implementadas.

Estas probas realízanse utilizando a aplicación en condicións normais, tanto nun uso correcto coma nun uso erróneo para comprobar que non se producen situacións non controladas. Este uso erróneo consiste en intentar facer accións non permitidas, deixar sen cubrir certa información á hora de gardar datos, cambiar a pantalla rapidamente ou antes de que remate unha acción, etc...

As validacións destas probas realizáronse a dous niveis: a comprobación dun funcionamento correcto da aplicación a nivel de usuario e a revisión dos datos introducidos, modificados e eliminados da base de datos.

\section{Probas de caixa negra}

do sistema completo


\section{Consola de Google}

\todo[inline]{ver TFG de Fernando Estévez}
versións beta, etc.

\section{Probas de rendemento}

para determinar a escalabilidade do sistema (número de usuarios, datos almacenados, ancho de bandas, etc.)


\section{Enquisa usuarios}

Enquisa de usuarios anónimos e de xestores de museos.
\todo[inline]{ver TFG de Gonzalo García}
