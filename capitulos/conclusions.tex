\chapter{Conclusións e traballo futuro}

Este capítulo está adicado a resumir as principais conclusión deste traballo asi como o traballo futuro.



\section{Conclusións}

Neste traballo proponse unha plataforma para a guía de museos que consta de dúas partes: un servidor e unha aplicación Android.
Ademáis, empregamos o SDK e os servidores de Situm Technologies S.L. para determinar a posición dentro de un edificio.

As principais características da nos plataforma son:
\begin{itemize}
 \item Modelo de datos flexible: puntos de interese (POI) e rutas de visita (lista de POI). Tempo estimado e outros datos de interese (imaxes, urls, valoracións, etc.)
 Diferentes categorías para adaptar a información mostrada ao tipo de usuario.
 \item Autenticación de usuarios para realizar tareas de mantemento e edición de contidos
 \item Usuarios anónimos para facilitar o uso da aplicación.
 \item Servidor na nube: procura escalabilidad, facilita o mantemento, permite axuste automático dos recursos empregados ao uso que se fai deles, ...
 \item Aplicación Android intuitiva e fácial de empregar para mostrar información do museo: mapas de cada planta, POIs, rutas de visita, etc.
 \item Localización Situm para calcular a posición dentro dos museos 
 (implica unha calibración previa de cada planta de cada edificio, por parte do administrador da plataforma ou do xestor de contidos do museo)
 \item Autenticación vía GoogleAuth para identificación dos usuarios privilegiados do noso sistema.
 \item Edición de datos (POI e rutas de visita) dentro da propia aplicación Android.
\end{itemize}




\section{Traballo futuro}

Este proyecto resolve os requisitos plantexados inicialmente, pero tamén abre novas vías e retos para o futuro:

\begin{itemize}
 \item Determinar automáticamente onde están máis tempo os visitantes no museo. Desta maneira se poderían axustar mellor os modelos: o tempo para ir dun POI a outro, 
 o tempo adicado a un POI, as rutas máis empregadas, os POIs máis visitas, o orde en que se visitan, etc.
 \item Outro punto interesante sería determinar, gracias ao seus movementos, qué tipos de usuarios visitan o usuarios.
   Con esta información, se podería ``adaptar'' a información que se mostra a cada usuario.
\end{itemize}



