\chapter{Implementación}

Neste capítulo se comentarán os aspectos máis relevantes da implementación da nosa plataforma.


\section{Servidor}

ASWS.
Autenticación con Google: Só facer mención á sección posterior

Spring en todo o proxecto

Exemplo de construción dos servizos web: Spring MVC
Exemplo de acceso a datos

\subsection{Transaccionalidade}
A xestión transaccionalidade impleméntase na capa Manager utilizando o framework Spring, grazas á súa librería spring-tx. A súa configuración e utilización é moi sinxela, tal e como se pode observar nos exemplos. No primeiro amósase a configuración que require nos ficheiros XML de Spring. Para indicar a transaccionalidade, utilizaremos etiquetas que permitan identificar o tipo de transacción que queremos que se aplique en cada método público do manager. Non todos os métodos provocan escrituras en base de datos, polo que non será necesario indicar o tipo de transaccionalidade que provoca desfacer cambios en todos eles. O resto marcaranse como de só lectura para non sobrecargar innecesariamente o sistema.

Configuración nos ficheiros XML:
\begin{lstlisting}
<tx:annotation-driven transaction-manager="txManager" />

<bean id="txManager" class="org.springframework.jdbc.datasource.DataSourceTransactionManager">
<property name="dataSource" ref="dataSource" />
</bean>
\end{lstlisting}

Exemplo de método transaccional:
\begin{lstlisting}
@Transactional
public Short gardarPercorrido(Percorrido percorrido, List<PuntoInterese> listaPuntoInterese)
\end{lstlisting}

Exemplo de método non transaccional:
\begin{lstlisting}
@Transactional(readOnly = true)
public Edificio getEdificio(Short idEdificio)
\end{lstlisting}


\section{Aplicación Android}

Como se solicitan permisos na aplicación

Exemplo de acceso a Situm

Exemplo de acceso ao servizo web

Exemplo de chamada a actividade e devolución do fluxo

Para publicar a aplicación na Play Store
https://developer.android.com/studio/publish/app-signing




\section{Autenticación}

https://developers.google.com/identity/sign-in/android/start-integrating


https://developers.google.com/android/guides/client-auth


Incluír o google-services.json dentro da aplicación Android para permitir a conexión