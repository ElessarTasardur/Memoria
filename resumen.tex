\chapter*{Resumo}

Poucos elementos tecnolóxicos entraron máis rápido na vida cotiá da xente que os teléfonos móbiles intelixentes, conseguindo facerse indispensábeis nun curto espazo de tempo. Son moitas as aplicacións que teñen no día a día das persoas, chegando a substituír por completo os elementos máis tradicionais cos que se levaban a cabo anteriormente. Entre estas aplicacións encóntrase a xeolocalización. Grazas a tecnoloxías coma o GPS (Global Positioning System) é posíbel situar un dispositivo en calquera punto do planeta cunha pequena marxe de erro, permitindo non só a localización dun elemento senón a indicación de rutas entre dous puntos.

Un dos principais problemas dos sistemas coma o GPS é a imposibilidade de seren utilizados en interiores. A localización en interiores presenta outro problema e é a precisión que se require para un correcto funcionamento, pois o erro asumíbel do GPS en exterior non sería viábel para os espazos interiores.

Nos últimos anos están aparecendo diversos sistemas de localización en interiores mediante a utilización de teléfonos móbiles intelixentes, algúns con gran precisión. Poderíanse describir como “sistemas GPS de interiores”. Non usan os sinais fornecidos polos satélites GPS, senón que utilizan outro tipo de sinais accesíbeis dende o teléfono móbil. Grazas a esta capacidade de localización en interiores é posíbel tamén a navegación.

A cantidade de información que pode haber ao redor dunha obra de arte pode ser abrumadora: dende as técnicas utilizadas para a creación dunha escultura, como o contexto histórico do artista ou o movemento artístico ao que pertence unha pintura. Cos métodos tradicionais, non sería posíbel mostrar toda a información dispoñíbel sobre unha obra nos museos.

O tempo dispoñíbel á hora de percorrer un museo de gran tamaño adoita ser un problema para os visitantes. Hai veces nas que o visitante desexa é admirar as obras de certos artistas ou movementos artísticos, o cal pode ser complicado se non posúe moita información.

\section*{Palabras clave}

Android, Localización en interiores, Situm Technologies, Guiado en museos.
