\chapter{Fundamentos tecnolóxicos}
Neste capítulo farase unha relación das tecnoloxías hardware e software nas que se basea o proxecto e outras que están en clara conexión. Tamén se tratará todo aquel software utilizado para a creación deste proxecto.


\section{Recursos hardware}
Para a realización deste proxecto empregáronse certos recursos hardware que pasaremos a detallar a continuación:


\subsection{Dispositivos (teléfonos) móbiles intelixentes}
Un dispositivo (teléfono) móbil intelixente é un ordenador de tamaño reducido con capacidades de comunicación pola rede telefónica a través de voz e datos. Están deseñados para realizar múltiples tarefas a través de software instalado no mesmo, xa sexa polo usuario ou fabricante, que se apoia nun sistema operativo que permite a utilización do hardware incluído no teléfono.

Os primeiros dispositivos foron creados a finais do século pasado mais non comezaron a estenderse ata esta década. Na actualidade é tan habitual o uso destes dispositivos que xa se contan por miles de millóns en todo o planeta.

Debido ao seu uso tan estendido é a opción escollida para ser a base do proxecto.

\subsubsection{One Plus 3T}

O dispositivo utilizado para as probas da aplicación Android foi o terminal One Plus 3T, coa versión 8.0 do sistema Android, un terminal de gama media que permitirá facerse unha idea do desempeño xeral da aplicación neste tipo de dispositivos.

As súas características pódense observar na táboa \ref{tab:tabCaracteristicasOnePlus}

\begin{table} [tbp]
	\begin{tabular}{|l|p{10cm}|}
		\hline 
		& \textbf{Características One Plus 3T} \\ 
		\hline 
		\textbf{Sistema Operativo} & Android 8.0 Oreo \\ 
		\hline 
		\textbf{Procesador} & Qualcomm Snapdragon 821 Quad Core, Kryo \\ 
		\hline 
		\textbf{Dimensións} & 152,7 x 74,7 x 7,35 mm \\ 
		\hline 
		\textbf{Peso} & 158 g \\ 
		\hline 
		\textbf{Pantalla} & 5,5 pulgadas cunha resolución 1080p Full HD (1920 x 1080 píxeles) \\ 
		\hline 
		\textbf{Memoria} & 6 GB \\ 
		\hline
		\textbf{Almacenamento} & 64 GB \\ 
		\hline 
	\end{tabular}
	\caption{Características One Plus 3T}
	\label{tab:tabCaracteristicasOnePlus}
\end{table}


\section{Recursos software}
Para a realización deste proxecto empregáronse certos recursos software. Dividídense en dous grupos atendendo á súa función dentro do proxecto, ben usados na propia elaboración do mesmo ou ben utilizados en tarefas de xestión e elaboración da memoria.

\subsection{Recursos para a elaboración do proxecto}
Estes recursos foron básicos na creación do sistema, xa sexa para a elaboración do código ou como plataforma para executar o noso códgio. A continuación pásase a explicar brevemente a súa función:

\subsubsection{SDK Situm}
Situm é unha empresa galega especializada no posicionamento en interiores. O seu sistema pódese implantar en calquera edificio cunha infraestrutura mínima, xa que incluso pode funcionar sen instalación de ningún tipo de dispositivos. Grazas a iso é posíbel un despregue inmediato no que só sería preciso a calibración do edificio que se pode levar a cabo en minutos. Situm proporciona un SDK (Software Development Kit - ou en galego, Kit de desenvolvemento de software) que permite utilizar o seu sistema tanto en Android coma iOS. Dispoñen dunha API pública na que poder revisar todos os elementos incluídos no SDK facilitando así o seu uso.


\subsubsection{Apache Tomcat}
Apache Tomcat é un contedor de servlets Java desenvolvido pola Apache Software Foundation. Implementa varias especificacións da Java Enterprise Edition e proporciona un entorno para servidores web HTTP no cal se pode executar código Java. Foi a opción escollida para o despregue dos servizos.


\subsubsection{Amazon Web Services}
Amazon Web Services (AWS) é unha plataforma de servizos na nube que ofrece gran potencia de cómputo, almacenamento para bases de datos, entrega de contido, entre outras funcionalidades aportando gran flexibilidade, un entorno escalábel e moi fiábel. Esta plataforma supón un gasto que depende da capacidade de cómputo requirida así coma o uso de rede. Utilizouse esta plataforma para lanzar o noso servidor e almacenar a base de datos.


\subsubsection{Eclipse}
Eclipse é un conxunto de ferramentas, proxectos e grupos de traballo en código aberto. Entre as ferramentas das que se compón, unha das máis utilizadas é o seu IDE (Integrated Development Environment - ou en galego, Entorno de desenvolvemento integrado), dispoñíbel para distintos sistemas operativos e amplamente utilizado para a programación en Java. Este foi o entorno que se utilizou para a elaboración dos servizos web.


\subsubsection{Java}
A plataforma Java é un entorno de computación creado por Sun Microsystems capaz de executar aplicacións desenvolvidas na linguaxe de programación Java, principalmente. A encargada de executar as aplicacións é a súa máquina virtual xunto cunha serie de bibliotecas estándar. Foi utilizada conxuntamente coa linguaxe de programación do mesmo nome para a creación dos servizos web.


\subsubsection{Android Studio}
É o IDE oficial para a programación en Android. Dende a súa saída substituíu ao Eclipse que era o anterior. Ao estar especificamente deseñado para o desenvolvemento de aplicacións en Android fai que a experiencia de programación para este sistema sexa moito máis rápida e cómoda. Foi o entorno utilizado para a programación da aplicación de Android.


\subsubsection{Postman}
Postman é un IDE destinado ao desenvolvemento de APIs. Permite a creación de peticións a APIs e a elaboración de tests de validación do seu comportamento entre outras características. Utilizouse para as probas da parte servidora.


\subsubsection{SQuirreL}
SQuirreL é un cliente de SQL que utiliza Java para se executar. A súa interface de acceso á base de datos é gráfica, o que permite ver a estrutura da mesma, así como navegar a través dos seus esquemas e táboas. A súa funcionalidade pode estenderse a través de plugins. Foi a aplicación utilizada para a xestión da base de datos.


\subsection{Recursos de xestión e elaboración da memoria}
A continuación faise unha relación dos recursos utilizados na xestión do proxecto ou na elaboración da memoria:

\subsubsection{Git}
Git é un sistema de control de versións que permite a coordinación de distintas persoas que realizan cambios sobre o mesmo traballo. Está orientado á rapidez, manter a integridade dos datos e soportar modelos de traballo distribuídos e non lineais. Como servidor para o control de versións utilizouse GitHub.


\subsubsection{LaTeX e TeXstudio}
LaTeX é un sistema de composición de textos orientado á creación de documentos cunha alta calidade tipográfica. Polas súas características e posibilidades, úsase especialmente na xeración de artigos e libros científicos que inclúen expresións matemáticas. Editoriais científicas de primeira liña utilizan este sitema debido á súa calidade.
TeXstudio é un IDE para a creación de documentos en LaTeX, cun deseño simple e agradábel, que integra numerosas ferramentas para a creación de documentos científicos. É de código aberto. Utilizouse este entorno para a elaboración da memoria.


\subsubsection{Inkscape}
É un editor de gráficos vectoriais, libre e de código aberto. Permite a conversión de ficheiros de imaxes clásicos en documentos de gráficos vectoriais. Utilizouse para a conversión das imaxes utilizadas na memoria.


\subsubsection{StarUML}
StarUML é un proxecto de código aberto que permite a creación de calquera tipo de diagrama UML. Utilizouse para a elaboración dos diagramas incluídos na memoria.
