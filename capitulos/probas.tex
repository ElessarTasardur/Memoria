\chapter{Probas}

Neste capítulo detallaremos as probas realizadas para validar o noso sistema, tanto ao finalizar cada sprint coma ao rematar o desenvolvemento. Tamén se terán en conta aquelas ferramentas utilizadas para localizar e ter un maior control sobre os problemas da aplicación.

\todo[inline]{ver TFG de Gonzalo García y Dylan Lema}


\section{Probas de validación de cada sprint}
Ao longo de todo o desenvolvemento do proxecto realizáronse probas ao finalizar cada sprint, ca fin de identificar posíbeis problemas introducidos sexa sobre as funcionalidades novas ou nas xa implementadas. A partir do terceiro sprint puidéronse realizar probas directamente sobre a aplicación pois nese momento xa se dispoñía dunha implementación básica que o permitía.

Estas probas realízanse utilizando a aplicación en condicións normais, tanto nun uso correcto coma nun uso erróneo para comprobar que non se producen situacións non controladas. Este uso erróneo consiste en intentar facer accións non permitidas, deixar sen cubrir certa información á hora de gardar datos, cambiar a pantalla rapidamente ou antes de que remate unha acción, etc...

As validacións destas probas realizáronse a dous niveis: a comprobación dun funcionamento correcto da aplicación a nivel de usuario e a revisión dos datos introducidos, modificados e eliminados da base de datos.


\section{Consola de Google}
Ao finalizar o terceiro sprint que xa supoñía unha aplicación usábel, aínda que limitada, procedeuse á publicación da aplicación en Google Play. Durante varios sprints foise publicando en modo Probas internas, sen permitir un acceso libre á mesma. Nesta modalidade permítese a proba dun número determinado de usuarios que se indican a través das súas contas de Google. Grazas a este sistema pódense recompilar os distintos erros que se produzan así como información sobre o dispositivo no que ocorren para poder dar unha solución con maior eficacia.
Despois do sexto sprint publicouse como beta aberta para poder acceder a un número maior de usuarios de proba. A publicación xera unha ligazón á aplicación que se pode compartir para que os usuarios a poidan descargar.


\todo[inline]{ver TFG de Fernando Estévez}
versións beta, etc.

\section{Probas de rendemento}

para determinar a escalabilidade do sistema (número de usuarios, datos almacenados, ancho de bandas, etc.)


\section{Enquisa usuarios}

Enquisa de usuarios anónimos e de xestores de museos.
\todo[inline]{ver TFG de Gonzalo García}
